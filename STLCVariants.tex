\documentclass[11pt]{article}

% style
\usepackage[a4paper,left=2.5cm,right=2.5cm,top=2.5cm,bottom=2.5cm]{geometry}

\usepackage{amsmath}
\usepackage{amssymb}
\usepackage{xcolor}
\usepackage{hyperref}
\usepackage[capitalise]{cleveref}

\usepackage{amsthm}
\newtheorem{lemma}{Lemma}
\newtheorem{theorem}{Theorem}
\newtheorem{corollary}{Corollary}

\usepackage[inference]{semantic} % for writing inference rules
\setnamespace{0pt}

\usepackage{enumitem}
\setlist{nolistsep}

\usepackage{empheq}
\newcommand{\boxedeq}[1]{\begin{empheq}[box=\fbox]{align*}#1\end{empheq}}

\let\t\texttt
\let\emptyset\varnothing
\let\to\rightarrow
\let\reduce\Rightarrow
\newcommand{\reduceM}{\Rightarrow^{*}}
\let\defas\triangleq
\newcommand{\Bool}{\t{Bool}}
\newcommand{\Some}[1]{\textsf{Some}(#1)}
\newcommand{\None}{\textsf{None}}
\newcommand{\kword}[1]{{\color{blue} \textsf{#1}}}
\newcommand{\True}{\kword{true}}
\newcommand{\False}{\kword{false}}
\newcommand{\If}{\kword{if}}
\newcommand{\Then}{\kword{then}}
\newcommand{\Else}{\kword{else}}

\title{\textbf{On the Meta Theory of Two STLC Variants}}
\author{Paul Zhu}
\date{\today}

\begin{document}

\maketitle

The \emph{simply typed lambda-calculus} (STLC, denoted by $\lambda_\to$) is a tiny core calculus for functional abstraction, embodying a simplest type system.
In this article, we are going to study the meta theory (more specifically, the type preservation property) of two STLC variants.
The original problem is exercise \t{stlc\_variation7} of this chapter of \emph{Software Foundations}:
\begin{center}
    \url{https://softwarefoundations.cis.upenn.edu/plf-current/StlcProp.html}.    
\end{center}
Formal proofs mechanized in Coq can be found here:
\begin{center}
    \url{https://github.com/paulzfm/STLCVariants/blob/main/STLCVariants.v}.
\end{center}

\section{The System}

Let's consider a STLC on Booleans, namely $\lambda_\to^{\Bool}$.
Syntax definition:
\begin{align*}
    \text{Term}~t &::= x & \text{(variable)} \\
    &\mid (t_1~t_2) & \text{($\lambda$-application)} \\
    &\mid (\lambda x:T.t) & \text{($\lambda$-abstraction)} \\
    &\mid \True & \text{(Boolean value true)} \\
    &\mid \False & \text{(Boolean value false)} \\
    &\mid (\If~t_1~\Then~t_2~\Else~t_3) & \text{(if-expression)} \\
    \text{(Simple) type}~T &::= \Bool & \text{(Boolean type is the only base type)} \\
    &\mid T_1\to T_2 & \text{(arrow type)}
\end{align*}

Notice that here we add three kinds of terms related to Booleans: Boolean value true, Boolean value false and if-expression.
Moreover, $\Bool$ is the sole base type in $\lambda_\to^{\Bool}$.
Typing rules for Booleans (other rules are standard):
\begin{align*}
    & \inference[T-True]{}{\Gamma \vdash \True:\Bool} \\
    & \inference[T-False]{}{\Gamma \vdash \False:\Bool} \\
    & \inference[T-If]{\Gamma \vdash t_1:\Bool & \Gamma \vdash t_2:T & \Gamma \vdash t_3:T}{\Gamma \vdash (\If~t_1~\Then~t_2~\Else~t_3):T}
\end{align*}

Evaluation rules for if-expression (other rules are standard):
\begin{align*}
    &\inference[E-If]{t_1 \reduce t_1'}{(\If~t_1~\Then~t_2~\Else~t_3) \reduce (\If~t_1'~\Then~t_2~\Else~t_3)} \\
    &\inference[E-IfTrue]{}{(\If~\True~\Then~t_2~\Else~t_3) \reduce t_2} \\
    &\inference[E-IfFalse]{}{(\If~\False~\Then~t_2~\Else~t_3) \reduce t_3}
\end{align*}

\section{Problem Statement}

Recall that the \emph{type preservation property} says, if $\vdash t:T$ and $t \reduce t'$, then $\vdash t':T$.

\paragraph{Q1}

Based upon the system $\lambda_\to^{\Bool}$, we add a funny typing rule
$$
    \inference[T-FunnyAbs]{}{\vdash (\lambda x:\Bool.t) : \Bool}
$$
and call it extended system (A).
Does the type preservation property holds for extended system (A)?

\paragraph{Q2}

Alternatively, suppose we change the funny typing rule into (since $\Gamma$ is a \emph{meta-variable}, it is regarded as arbitrary)
$$
    \inference[T-FunnyAbs']{}{\Gamma \vdash (\lambda x:\Bool.t) : \Bool}
$$
and call it extended system (B).
Does the type preservation property holds for extended system (B)?

\paragraph{Remark} Before looking at the answers (on the continuing pages), I strongly recommend you to think about the two questions yourself and make a smart guess.

\newpage

\section{Answer to Q1}

As you may guess, the type preservation property does \textbf{not} hold. Let's see a counterexample.
Given terms
\begin{align*}
    t &\defas (\lambda x:\Bool.\lambda y:\Bool.x)~(\lambda x:\Bool.z) \\
    t' &\defas \lambda y:\Bool.\lambda x:\Bool.z
\end{align*}
We see that $\vdash t: \Bool\to\Bool$ and $t \reduce t'$.
However, $t'$ is not typable under empty context, since the rule T-FunnyAbs is \textbf{only} applicable when the typing context is \textbf{empty}.
But in typing $t'$, the subterm $\lambda x:\Bool.z$ is typing under $\emptyset,y:\Bool$, which is nonempty.

Some of you may raise other ``counterexamples'', which are in fact wrong.
For instance, one may construct the following two terms
\begin{align*}
    t &\defas (\lambda x:\Bool.\lambda y:\Bool.x)~\True \\
    t' &\defas \lambda y:\Bool.\True
\end{align*}
which makes $\vdash t: \Bool\to\Bool$ and $t \reduce t'$ hold.
Then, the student say, well, since $\vdash t': \Bool$ by the rule T-FunnyAbs, and of course $\Bool$ differs from $\Bool\to\Bool$, then the preservation property violates.
If you read the explain twice, you will soon see this is in fact not a valid counterexample!
Using the normal typing rules, we could easily show that $\vdash t': \Bool\to\Bool$, meaning the type of $t'$ is preserved.
In general, a valid counterexample for the preservation property must point out, (1) either $t'$ is ill-typed, or (2) $t'$ \textbf{cannot} have the original type, rather than it \textbf{can} have another type.

\section{Answer to Q2}

Note that the funny typing rule is now applicable under \textbf{any} typing context $\Gamma$.
We realize the above counterexample could \textbf{not} demonstrate the invalidness of the preservation property.
We will see that the property holds for this extended system.
To show this, let's first review the proof techniques for the original system.

\subsection{Preservation of the Original System}

Recall that in the original system, $\True$, $\False$ and abstractions are \emph{values}.
We write $val(t)$ to denote a term $t$ is a value.
We specify the following evaluation rules, expressed in the style of small-step operational semantics:
\begin{align*}
    &\inference[E-App1]{t_1 \reduce t_1'}{t_1~t_2 \reduce t_1'~t_2}
    &\inference[E-App2]{val(v_1) & t_2 \reduce t_2'}{v_1~t_2 \reduce v_1~t_2'} \\
    &\inference[E-$\beta$]{val(v)}{(\lambda x:T.t)~v \reduce t[x:=v]} \\
    &\inference[E-If]{t_1 \reduce t_1'}{(\If~t_1~\Then~t_2~\Else~t_3) \reduce (\If~t_1'~\Then~t_2~\Else~t_3)} \\
    &\inference[E-IfTrue]{}{(\If~\True~\Then~t_2~\Else~t_3) \reduce t_2} \\
    &\inference[E-IfFalse]{}{(\If~\False~\Then~t_2~\Else~t_3) \reduce t_3}
\end{align*}

In the $\beta$-reduction rule E-$\beta$, the substitution operation $t[x:=s]$, meaning substituting a term $s$ for a variable $x$ in a term $t$, is inductively defined as follows:
\begin{align*}
    x[x:=s] &= s \\
    y[x:=s] &= y & \text{if}~x\not=y \\
    (\lambda x:T.t)[x:=s] &= \lambda x:T.t \\
    (\lambda y:T.t)[x:=s] &= \lambda y:T.(t[x:=s]) & \text{if}~x\not=y \\
    (t_1~t_2)[x:=s] &= (t_1[x:=s]~t_2[x:=s]) \\
    \True[x:=s] &= \True \\
    \False[x:=s] &= \False \\
    (\If~t_1~\Then~t_2~\Else~t_3)[x:=s] &= \If~t_1[x:=s]~\Then~t_2[x:=s]~\Else~t_3[x:=s]
\end{align*}

By the way, we denote the set of \emph{free variables} of the term $t$ as $FV(t)$, inductively defined as follows:
\begin{align*}
    FV(x) &= \{x\} \\
    FV(t_1~t_2) &= FV(t_1) \cup FV(t_2) \\
    FV(\lambda x:T.t) &= FV(t) \setminus \{x\} \\
    FV(\True) &= \emptyset \\
    FV(\False) &= \emptyset \\
    FV(\If~t_1~\Then~t_2~\Else~t_3) &= FV(t_1) \cup FV(t_2) \cup FV(t_3)
\end{align*}
A term is \emph{closed} if it contains no free variables. Otherwise, it is \emph{open}.

A \emph{typing context} consists of a sequence of bindings mapping variables to their types, inductively defined as
$$\Gamma ::= \emptyset \mid \Gamma,x:T$$
We implement it as a \emph{partial map} from variables to \emph{optional types} (in Haskell, the \t{Maybe} monad) -- either $\None$ for nothing, or $\Some{T}$ for type $T$.
As a convention, we write $x \not\in dom(\Gamma)$ if $\Gamma(x)=\None$, and use the substitution notion $\Gamma[x:=T]$ for a context update $\Gamma,x:T$.
Typing rules are given as the following:
\begin{align*}
    & \inference[T-Var]{\Gamma(x)=\Some{T}}{\Gamma \vdash x:T}
    & \inference[T-Abs]{\Gamma,x:T \vdash t:T'}{\Gamma \vdash (\lambda x:T.t):T \to T'} \\
    & \inference[T-App]{\Gamma \vdash t_1:T \to T' & \Gamma \vdash t_2:T}{\Gamma \vdash (t_1~t_2):T'} \\
    & \inference[T-True]{}{\Gamma \vdash \True:\Bool}
    & \inference[T-False]{}{\Gamma \vdash \False:\Bool} \\
    & \inference[T-If]{\Gamma \vdash t_1:\Bool & \Gamma \vdash t_2:T & \Gamma \vdash t_3:T}{\Gamma \vdash (\If~t_1~\Then~t_2~\Else~t_3):T}
\end{align*}
A term $t$ is \emph{well-typed} if it is typable under the empty typing context, i.e., $\vdash t:T$ for some type $T$.
Otherwise, it is \emph{ill-typed}.

So far, we have set up all the required notations.
To show the type preservation theorem, we will need to develop some technical machinery for reasoning about variables
and substitution.
Working from top to bottom (from the high-level property we are actually interested in to the lowest-level technical lemmas that are needed by various cases of the more interesting proofs), the story goes like this:
\begin{itemize}
    \item The \emph{preservation theorem} is proved by induction on a typing derivation.
    The one case that is significantly different is the one for the $\beta$-reduction rule, whose definition uses the substitution operation. 
    To see that this step preserves typing, we need to know that the substitution itself does. So we prove a \ldots
    \item \emph{substitution lemma}, stating that substituting a (closed) term $s$ for a variable $x$ in a term $t$ preserves the type of $t$.
    The proof goes by induction on the form of $t$ and requires looking at all the different cases in the definition of substitution.
    This time, the tricky cases are the ones for variables and for abstractions. 
    In both, we discover that we need to take a term $s$ that has been shown to be well-typed in some context $\Gamma$ and consider the same term $s$ in a slightly different context $\Gamma'$.  For this we prove a \ldots
    \item \emph{context invariance} lemma, showing that typing is preserved under ``inessential changes'' to the context $\Gamma$ -- in particular, changes that do not affect any of the free variables of the term.
\end{itemize}

To make both Coq and mathematicians happy, we have to start from the bottom and reach the top step by step.
First of all, let's formalize and show the context invariance lemma:
\begin{lemma}\label{thm:ori:inv}
    (Context invariance) Suppose (1) $\Gamma \vdash t:T$, and (2) for every $x\in FV(t)$, $\Gamma(x)=\Gamma'(x)$, then $\Gamma' \vdash t:T$.
\end{lemma}
\begin{proof}
    By induction on the typing judgment $\Gamma \vdash t:T$.
    \begin{itemize}
        \item For the T-Var case $\Gamma \vdash x:T$, we have $\Gamma(x)=\Some{T}$.
        Since $x \in FV(x)$, we must also have $\Gamma(x) = \Gamma'(x)$, which implies $\Gamma'(x)=\Some{T}$,
        and thus $\Gamma' \vdash x:T$ holds by an application of T-Var.
        \item For the T-Abs case $\Gamma \vdash (\lambda x:U.t):U \to T$, we have $\Gamma,x:U \vdash t:T$.
        Notice that $FV(\lambda x:U.t)=FV(t) \setminus \{x\}$.
        Since $\Gamma(y)=\Gamma'(y)$ for every $y\in FV(\lambda x:U.t)$, 
        \begin{equation}\label{eq:inv}
            (\Gamma,x:U)(y)=(\Gamma',x:U)(y)    
        \end{equation}
        holds for every $y\in FV(t)$ whenever $y \not=x$.
        Even when $y = x$, it is trivially seen that \cref{eq:inv} still holds.
        Thus by inductive hypothesis, we have $\Gamma',x:U \vdash t:T$, which implies that $\Gamma' \vdash (\lambda x:U.t): U \to T$ by an application of T-Abs.
        \item For the T-App case $\Gamma \vdash (t_1~t_2):T$, we know that $\Gamma \vdash t_1:T_1 \to T$ for some $T_1$ and $\Gamma \vdash t_2:T_1$.
        Notice that $FV(t_1~t_2)=FV(t_1) \cup FV(t_2)$.
        We see that $\Gamma(y)=\Gamma'(y)$ for every $y \in FV(t_1)$ and every $y \in FV(t_2)$.
        Thus by inductive hypothesis, we have $\Gamma' \vdash t_1:T_1 \to T$, as well as $\Gamma' \vdash t_2:T_1$, which imply that $\Gamma' \vdash (t_1~t_2): T$ by an application of T-App.
        \item The cases for T-True and T-False are immediate, as their typing judgments hold under any typing context.
        \item Finally, the T-If case is similar to the T-App case, except the inductive hypothesis needs be used three times, once for each of the three premises of T-If.
    \end{itemize}
\end{proof}

Before we move to a useful corollary of \cref{thm:ori:inv}, let's first show a simple fact -- well-typed terms must be closed, as stated in the following lemma:
\begin{lemma}\label{thm:ori:closed}
    (Well-typed terms are closed) If $\vdash t:T$, then $FV(t)=\emptyset$.
\end{lemma}
\begin{proof}
    We first show that if $\Gamma \vdash t:T$, then for every $x \in FV(t)$, $x \in dom(\Gamma)$,
    by induction on the typing judgment $\Gamma \vdash t:T$:
    \begin{itemize}
        \item For the T-Var case $\Gamma \vdash x:T$, we have $\Gamma(x)=\Some{T}$.
        Since $FV(x)=\{x\}$, the conclusion holds.
        \item For the T-Abs case $\Gamma \vdash (\lambda x:U.t'):U \to T'$, we have $\Gamma,x:U \vdash t':T'$.
        For every $y \in FV(t) = FV(t') \setminus \{x\}$, we know $y \in FV(t')$ and thus $y \in dom(\Gamma,x:U)$ by inductive hypothesis.
        Moreover, since $y \not=x$, we must have $y \in dom(\Gamma)$.
        \item For the T-App case $\Gamma \vdash (t_1~t_2):T$, we have $\Gamma \vdash t_1:T_1 \to T$ for some $T_1$ and $\Gamma \vdash t_2:T_1$.
        For every $y \in FV(t)$, i.e., $y \in FV(t_1)$ or $y \in FV(t_2)$, in either case, we could conclude that $y \in dom(\Gamma)$ by inductive hypothesis.
        \item For the T-True and T-False cases, $FV(t)=\emptyset$ by definition and thus the conclusion trivially holds.
        \item For the T-If case $\vdash (\If~t_1~\Then~t_2~\Else~t_3):T$, we have $\vdash t_1:\Bool$, $\vdash t_2:T$ and $\vdash t_3:T$.
        For every $y \in FV(t)$, i.e., $y \in FV(t_1)$ or $y \in FV(t_2)$ or $y \in FV(t_3)$, in either case, we could conclude that $y \in dom(\Gamma)$ by inductive hypothesis.
    \end{itemize}
    Now, let $t$ be a well-typed term, and suppose there exists some variable $x\in FV(t)$, then according to the above, $x \in dom(\emptyset)$, which is a contradiction!
    Therefore, it must be the case that $FV(t)=\emptyset$.
\end{proof}

Now, we state the corollary of \cref{thm:ori:inv}, which says, a well-typed term is typable under whatever typing context, and the typing is unique:
\begin{corollary}\label{thm:ori:inv-coro}
    If $\vdash t:T$, then $\Gamma \vdash t:T$ for any $\Gamma$.
\end{corollary}
\begin{proof}
    In \cref{thm:ori:inv}, pick $\Gamma$ as $\emptyset$ and leave $\Gamma'$ arbitrary.
    Apparently, we have $\vdash t:T$. By \cref{thm:ori:closed}, $FV(t)=\emptyset$.
    Thus, condition (2) of \cref{thm:ori:inv} holds forever.
    By \cref{thm:ori:inv}, we conclude that $\Gamma' \vdash t:T$ for any $\Gamma'$.
\end{proof}

Next, we come to the substitution lemma, which says substituting a closed term $s$ for a variable $x$ in a term $t$ preserves the type of $t$:
\begin{lemma}\label{thm:ori:subst}
    (Substitution preserves typing) If $\Gamma,x:U \vdash t:T$ and $\vdash s:U$, then $$\Gamma \vdash t[x:=s]:T.$$
\end{lemma}
\begin{proof}
    By induction on the form of the term $t$.
    \begin{itemize}
        \item Case $t$ is a variable, say $y$.
        \begin{itemize}
            \item Suppose that $y=x$. By inversion on $\Gamma,x:U \vdash x:T$, we have $U=T$ and thus $\vdash s:T$.
            By \cref{thm:ori:inv-coro}, we have
            $$\Gamma \vdash s:T.$$
            The above is indeed what we need to show, once we realize that $x[x:=s]=s$.
            \item Otherwise, $y\not=x$.
            By inversion on $\Gamma,x:U \vdash y:T$, we have $\Gamma(y)=\Some{T}$, which implies
            $$\Gamma \vdash y:T$$
            by an application of T-Var.
            The above is indeed what we need to show, once we realize that $y[x:=s]=y$.
        \end{itemize}
        \item Case $t$ is an abstraction, say $(\lambda y:S.t')$.
        By inversion on $\Gamma,x:U \vdash (\lambda y:S.t'):T$, we have
        \begin{equation}\label{eq:abs:H}
            \Gamma,x:U,y:S \vdash t':T'
        \end{equation}
        such that $T=S \to T'$.
        \begin{itemize}
            \item Suppose that $y=x$. \cref{eq:abs:H} is simply
            $$\Gamma,x:S \vdash t':T'.$$
            By an application of T-Abs, $$\Gamma \vdash (\lambda x:S.t'):S \to T'.$$
            The above is indeed what we need to show, once we realize that $(\lambda x:S.t')[x:=s]=\lambda x:S.t'$.
            \item Otherwise, $y\not=x$. \cref{eq:abs:H} is simply
            $$\Gamma,y:S,x:U \vdash t':T'.$$
            By inductive hypothesis, $$\Gamma,y:S \vdash t'[x:=s]:T'.$$
            By an application of T-Abs, $$\Gamma \vdash \lambda y:S.(t'[x:=s]):S \to T'.$$
            The above is indeed what we need to show, once we realize that $(\lambda y:S.t')[x:=s]=\lambda y:S.(t'[x:=s])$.
        \end{itemize}
        \item Case $t$ is an application, say $(t_1~t_2)$.
        By inversion on $\Gamma,x:U \vdash (t_1~t_2):T$, we have
        $$\Gamma,x:U \vdash t_1:U' \to T~\text{and}~\Gamma,x:U \vdash t_2:U'$$
        for some $U'$.
        By inductive hypothesis, $$\Gamma \vdash t_1[x:=s]:U' \to T~\text{and}~\Gamma \vdash t_2[x:=s]:U',$$
        which imply $$\Gamma \vdash t_1[x:=s]~t_2[x:=s]:T$$
        by an application of T-App.
        The above is indeed what we need to show, once we realize that $(t_1~t_2)[x:=s]=t_1[x:=s]~t_2[x:=s]$.
        \item Case $t$ is $\True$ or $\False$, no substitution is really happened and thus the conclusion trivially holds.
        \item Finally, case $t$ is an if-expression, the situation is quite similar to the application case, except that the inductive hypothesis needs be used three times, once for each of the three premises of T-If.
    \end{itemize}
\end{proof}

Ladies and gentlemen, finally comes our type preservation theorem!
\begin{theorem}\label{thm:ori:main}
    (Preservation) If $\vdash t:T$ and $t \reduce t'$, then $\vdash t':T$.
\end{theorem}
\begin{proof}
    By induction on the typing judgment $\vdash t:T$.
    We should omit the cases for T-Abs, T-True and T-False, where the term $t$ is already a value.
    \begin{itemize}
        \item The T-Var case $\vdash x:T$ is impossible, as $x \not\in dom(\emptyset)$ for any $x$.
        \item For the T-App case $\vdash (t_1~t_2):T$, we have $\vdash t_1:T_1 \to T$ for some $T_1$ and $\vdash t_2:T_1$.
        We now make an inversion on the evaluation relation $t \reduce t'$. Only three rules are applicable:
        \begin{itemize}
            \item Case E-$\beta$.
            The term $t_1$ must be in form $\lambda x:T_1.t_3$, such that $\emptyset,x:T_1 \vdash t_3:T$,
            and $t'=t_3[x:=t_2]$.
            By \cref{thm:ori:subst} (let $\Gamma$ be $\emptyset$), $\vdash t_3[x:=t_2]:T$, i.e., $\vdash t':T$.
            \item Case E-App1. We have $t_1 \reduce t_1'$ and $t_1~t_2 \reduce t_1'~t_2$.
            By inductive hypothesis, $\vdash t_1':T_1 \to T$.
            Since $\vdash t_2:T_1$, by an application of T-App, $\vdash (t_1'~t_2):T$.
            \item Case E-App2. Similar to case E-App1.
        \end{itemize}
        \item For the T-If case $\vdash (\If~t_1~\Then~t_2~\Else~t_3):T$, we have $\vdash t_1:\Bool$, $\vdash t_2:T$ and $\vdash t_3:T$.
        Again, we make an inversion on the evaluation relation $t \reduce t'$.
        All three cases are similar to case E-App1.
    \end{itemize}
\end{proof}

\subsection{Preservation of the Extended System}

We now move onto the proof for extended system (B). Recall that we have one more funny typing rule:
$$
    \inference[T-FunnyAbs']{}{\Gamma \vdash (\lambda x:\Bool.t) : \Bool}
$$

If we copy all the previous proof mechanism, we will soon realize that \cref{thm:ori:closed} does not hold anymore!
As a counterexample, the originally open (not closed) term $\lambda x:\Bool.y$ is now well-typed, thanks to its ``argument'' having type \Bool.
However, \cref{thm:ori:inv-coro} still holds for the extended system.
We convince ourselves by the following intuition: suppose that $\vdash t:T$.
If $t$ is a closed term, then obviously $\Gamma \vdash t:T$ holds for any typing context $\Gamma$.
Otherwise, in typing the term $t$, the funny typing rule T-FunnyAbs' must have been applied.
Since this rule does not depend on types of free variables -- it only requires the bound variable to be a boolean -- we see that $\Gamma \vdash t:T$ still holds for any $\Gamma$.

To show the corollary, we need another form of the context invariance lemma.
The original form (\cref{thm:ori:inv-coro}) is in fact too ``strict'' for the extended system.
To make $\Gamma' \vdash t:T$ derivable from $\Gamma \vdash t:T$, we only require the free variables of $t$ that appears in $\Gamma$ to have the identical type in $\Gamma'$.
For those that does not appear in $\Gamma$, they could be whatever in $\Gamma'$, since they are unused in typing $t$ under $\Gamma'$. Formally,
\begin{lemma}\label{thm:ext:inv}
    (Context invariance) Suppose that (1) $\Gamma \vdash t:T$, and (2) for every $x\in FV(t)$ such that $x\in dom(\Gamma)$, $\Gamma(x)=\Gamma'(x)$, then $\Gamma' \vdash t:T$.
\end{lemma}
\begin{proof}
    By induction on the typing judgment $\Gamma \vdash t:T$.
    The original six cases all carry over here.
    The T-FunnyAbs' case is also immediate, as the typing judgment holds under any typing context.
\end{proof}

Now the wanted corollary is immediate:
\begin{corollary}\label{thm:ext:inv-coro}
    If $\vdash t:T$, then $\Gamma \vdash t:T$ for any $\Gamma$.
\end{corollary}
\begin{proof}
    In \cref{thm:ext:inv}, pick $\Gamma$ as $\emptyset$ and leave $\Gamma'$ arbitrary.
    Apparently, condition (2) of \cref{thm:ext:inv} holds forever, as $dom(\emptyset)=\emptyset$.
    By \cref{thm:ext:inv}, we conclude that $\Gamma' \vdash t:T$ for any $\Gamma'$.
\end{proof}

The proof of \cref{thm:ori:subst} and \cref{thm:ori:main} all carry over here, with the help of \cref{thm:ext:inv-coro}.
To sum up, type preservation theorem still holds for extended system (B).

\end{document}